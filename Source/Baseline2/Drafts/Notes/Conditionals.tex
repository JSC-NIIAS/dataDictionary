\documentclass[parskip=half,paper=a4,DIV=13]{scrartcl}


\usepackage[utf8]{inputenc}
\usepackage{pgfplots}
\usepackage{amsmath}
\usepackage{amsthm}
\usepackage{tikz}
\usepackage{verbatim}
\usepackage{amssymb}
\usepackage{breqn}
\usepackage{graphicx}
\usepackage{color}
\usepackage{fancyhdr}
\usepackage{listings}
\usepackage{float}
\usepackage{stmaryrd}
\usepackage{cancel}

\setlength{\headheight}{22pt}
\pagestyle{fancy}

\lstdefinestyle{customc}{
  belowcaptionskip=1\baselineskip,
  breaklines=true,
  frame=L,
  xleftmargin=\parindent,
  language=C,
  showstringspaces=false,
  basicstyle=\footnotesize\ttfamily,
  keywordstyle=\bfseries\color{green!40!black},
  commentstyle=\itshape\color{purple!40!black},
  identifierstyle=\color{blue},
  stringstyle=\color{orange},
}

\lstdefinestyle{customasm}{
  belowcaptionskip=1\baselineskip,
  frame=L,
  xleftmargin=\parindent,
  language=[x86masm]Assembler,
  basicstyle=\footnotesize\ttfamily,
  commentstyle=\itshape\color{purple!40!black},
}
\lstset{escapechar=@,style=customc}



\title{Decoder and Encoder for Dynamic C-Structures}

\author{Timon Lapawczyk}

\begin{document}

\maketitle

\section{Introduction}

The Decoder and Encoder functions for all static packets now make use of the Bitstream\_Write
and Bitstream\_Read functions. Therefore one does not have to calculate the bitpositions himself
when it comes to the C-Code.

However this does not apply to the formal specification.
While checking a predicate like Equal-Bits, frama-c can not access the states within the function,
to obtain the current bitpos.

\begin{lstlisting}[mathescape]
predicate EqualBits(Bitstream* stream, integer pos, Adhesion_Factor* p) =
    EqualBits(stream, pos,       pos + 2,   p->Q_DIR)          &&
    EqualBits(stream, pos + 2,   pos + 15,  p->L_PACKET)       &&
    EqualBits(stream, pos + 15,  pos + 17,  p->Q_SCALE)        &&
    EqualBits(stream, pos + 17,  pos + 32,  p->D_ADHESION)     &&
    EqualBits(stream, pos + 32,  pos + 47,  p->L_ADHESION)     &&
    EqualBits(stream, pos + 47,  pos + 48,  p->M_ADHESION);
\end{lstlisting}

In the next step we focus on the C-Code and leave the formal specification aside.
Hence we can for now ignore the problem with the predicates and focus on other problems.

\section{Bitstream\_Write and Bitstream\_Read}

With the switch to the Bitstream functions the current bitpos is saved within the Bitstream
structure and rewritten during the Bitstream\_Read and Bitstream\_Write calls.

That makes the Decoder and Encoder functions a lot cleaner.

In the example below all read calls are independent from one another.

\begin{lstlisting}[mathescape]
int Infill_location_reference_Decoder(Bitstream* stream, Infill_location_reference* p)
{
    if (NormalBitstream(stream, INFILL_LOCATION_REFERENCE_BITSIZE))
    {
        uint8_t* addr = stream->addr;
	const uint32_t size = stream->size;
	const uint32_t pos = stream->bitpos;

	p->Q_DIR           = Bitstream_Read(stream, 2);
	p->L_PACKET        = Bitstream_Read(stream, 13);
	p->Q_NEWCOUNTRY    = Bitstream_Read(stream, 1);
	p->NID_C           = Bitstream_Read(stream, 10);
	p->NID_BG          = Bitstream_Read(stream, 14);

	return 1;
    }
    else
    {
        return 0;
    }
}
\end{lstlisting}

\section{Optional variables}

As it happens, the example above contains an optional variable.

\begin{itemize}

\item
The value for NID\_C is only read if Q\_NEWCOUNTRY has the value 1.
\item
The dependency of the value Q\_NEWCOUNTRY can be realized via an if clause.
\item
Since all Bitstream\_Read calls are independent the bitpos for NID\_BG remains unchanged
if NID\_C is not read and adds up automatically if NID\_C is read.

\end{itemize}

The NID\_C line from above changes into

\begin{lstlisting}[mathescape]
if (p->Q_NEWCOUNTRY == 1)
{
    p->NID_C         = Bitstream_Read(stream, 10);
}
\end{lstlisting}

The first question at hand is how do we calculate the value for INFILL\_LOCATION\_REFERENCE\_BITSIZE
to check NormalBitstream?

Until now this value has been set in the Infill\_location\_reference.h file, as the sum of the bitsizes
of all variables.
Now the actual bitsize may depend on one or more values, read during runtime.
There is no access to these values when deciding, whether the stream is long enought to read the whole packet from.

To maintain some error handling we do not change this value.
We can still assure to capture all error cases, where there are not enought bits to read from, in the stream.
However we can not assure anymore that a return of 0 means that there were not enought bits.
A not read optional variable might have also been the cause of the error.

\section{Bitsize in Decoder Branch}

Maybe we can recycle some of our thoughts on the header files. The L\_PACKET value equals the number
of bits, that are actually transferred and therefore the neccessary length of the stream.
If we read the L\_PACKET bits together with the NID\_PACKET value in the decoder branch,
we can hand it on to the decoder.

\end{document}
