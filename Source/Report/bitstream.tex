

\chapter{The \inl{Bitstream} Layer}

Conceptually, a \emph{bit stream}~$b$ is a sequence~$b_0,\ldots,b_{n-1}$ of $n$~bits.
Since the \isoc programming language does not allow to directly declare bit sequences
we represent bit streams as \emph{byte arrays}, that is as arrays of type \inl{uint8_t}.

\section{The Type \bitstream and Related Functions}
\subsection{The Function \bitstreamread}
\subsection{The Function \inl{Bitstream_Write}}
\subsection{The Function \inl{NormalBitstream}}

\section{Predicates}
\subsection{\inl{Readable} and \inl{Writeable}}
\subsection{\inl{Invariant} and \inl{Normal}}
\subsection{\inl{EqualBits} and \inl{Unchanged}}
\subsection{\inl{UpperBitsNotSet}}

\section{Formal Specification}
\subsection{The Function \bitstreamread}
\subsection{The Function \inl{Bitstream_Write}}
\subsection{The Function \inl{NormalBitstream}}

\section{Implementation}
\subsection{The Function \bitstreamread}
\subsection{The Function \inl{Bitstream_Write}}
\subsection{The Function \inl{NormalBitstream}}

\section{Formal Verification}
\subsection{\inl{Bitstream_ReadThenWrite}}
\subsection{\inl{Bitstream_WriteThenRead}}

