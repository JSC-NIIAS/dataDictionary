
\documentclass[paper=a4,12pt,DIV16,BCOR8mm,fleqn,twoside]{scrartcl}

\usepackage[utf8]{inputenc}
\usepackage[T1]{fontenc}
\usepackage{lmodern}
\usepackage[english]{babel}

\usepackage[pdftex]{graphicx}
\usepackage{xspace}
\usepackage{textcomp}

% needed for FloatBarrier
\usepackage[section]{placeins}

\usepackage{verbatim}
\usepackage[nomargin,inline,draft]{fixme}

\usepackage{comment}

%\usepackage[firstpage]{draftwatermark}
%\SetWatermarkScale{4}

\usepackage{amsmath}
\usepackage{amssymb}
\usepackage{amsthm}
\usepackage{txfonts}


\usepackage{booktabs}
\usepackage{rotating}
\usepackage{multicol}
\usepackage{multirow}
\usepackage{multicol}

\newcommand{\sturz}[2]{\rotatebox{90}{\parbox{#2}{\center #1}}}

\usepackage{listings}
%Defining C-Code Environment

\usepackage{courier} 
\usepackage{listings}
\usepackage{color} 

% fix bug with listing under texlive-2014
% see https://lists.debian.org/debian-tex-maint/2014/06/msg00209.html

\makeatletter
\renewcommand\lstinline[1][]{%
  \leavevmode\bgroup % \hbox\bgroup --> \bgroup
  \def\lst@boxpos{b}%
  \lsthk@PreSet\lstset{flexiblecolumns,#1}%
  \lsthk@TextStyle
  \ifnum\iffalse{\fi`}=\z@\fi
  \@ifnextchar\bgroup{%
  \ifnum`{=\z@}\fi%
  \afterassignment\lst@InlineG \let\@let@token}{%
  \ifnum`{=\z@}\fi\lstinline@}}
\makeatother

% 1.3.10 jb:
% commented-out unused colors and styles,
% changed color names to logical ones

%\definecolor{darkred}		{rgb}{0.60,0.00,0.00}
\definecolor{coACSLBehavior}	{rgb}{0.30,0.00,0.00}
\definecolor{coASCL}		{rgb}{0.00,0.10,0.00}
\definecolor{coASCLKeyword}	{rgb}{0.00,0.10,0.10}
\definecolor{darkgreen}		{rgb}{0.00,0.40,0.00}
\definecolor{darkblue}		{rgb}{0.00,0.00,0.60}
\definecolor{coCKeyword}	{rgb}{0.00,0.00,0.60}
%\definecolor{red}		{rgb}{0.98,0.00,0.00}
%\definecolor{lightblue}        {rgb}{0.60,0.80,1.00}
%\definecolor{lightred}		{rgb}{1.00,0.60,0.60}

\lstdefinestyle{acsl-block}
{
  emph=[1]{assert, assumes, assigns, axiom, axiomatic, decreases, ensures,
                 ghost, invariant, lemma, logic, loop, predicate,
		 reads, requires, variant},
  emphstyle=[1]{\bfseries\color{coASCLKeyword}},
  emph=[2]{behavior, behaviors, complete, disjoint, for:},
  emphstyle=[2]{\bfseries\color{coACSLBehavior}},
  emph=[3]{typedef, int, char, integer, real, bool, size_type, value_type},
  emphstyle=[3]{\bfseries\color{coCKeyword}},
  escapeinside={//`}{`//},
  morecomment=*[l][\color{coASCL}]{//@},
  morecomment=*[s][\color{coASCL}]{/*@}{*/},
  moredelim=*[is][\bfseries]{|*}{*|}
  %emphstyle=[3]{\ttfamily}
  }

\lstdefinestyle{func-decl}
{
  emph=[1]{assert, assumes, assigns, axiom, axiomatic, decreases, ensures,
                 ghost, invariant, lemma, logic, loop, predicate,
		 reads, requires, variant},
  emphstyle=[1]{\bfseries\color{coASCLKeyword}},
  emph=[2]{behavior, behaviors, complete, disjoint, for:},
  emphstyle=[2]{\bfseries\color{coACSLBehavior}},
  emph=[3]{integer, real, size_type, value_type},
  emphstyle=[3]{\bfseries\color{coCKeyword}},
  escapeinside={//`}{`//},
  morecomment=*[l][\color{coASCL}]{//@},
  morecomment=*[s][\color{coASCL}]{/*@}{*/},
  moredelim=*[is][\bfseries]{|*}{*|},
     frame=none,
     numbers=none
  %emphstyle=[3]{\ttfamily}
}

\lstdefinestyle{acsl-inline}
{
  emph=[1]{assert, assumes, axiom, axiomatic, decreases, ensures,
           ghost, invariant, lemma, logic, loop,
           predicate, reads, requires, return, variant },
  emphstyle=[1]{\bfseries\color{coASCLKeyword}},
  emph=[2]{behavior, behaviors, complete, disjoint, for:},
  emphstyle=[2]{\bfseries\color{coACSLBehavior}},
  emph=[3]{typedef, int, char, integer, real, bool, size_type, value_type},
  emphstyle=[3]{\bfseries\color{coCKeyword}},
  morecomment=*[l][\color{coASCL}]{//@},
  morecomment=*[s][\color{coASCL}]{/*@}{*/},
  moredelim=*[is][\bfseries]{|*}{*|}
}

\lstdefinestyle{inline}
{
  basicstyle = \ttfamily\small\color{coASCL},
  keywordstyle = \ttfamily\small\color{coASCL},
  stringstyle=\color{coASCL},
  style=acsl-inline,
  breaklines= false
}

\lstset
{%
  language=C++,
  defaultdialect=ansi,
  basicstyle=\small\ttfamily,
  commentstyle=\small\color{darkgreen},
  keywordstyle=\small\bfseries\color{darkblue},
  stringstyle=\small\color{darkgreen},
  tabsize = 2,
  showspaces=false,
  showtabs=false,
  columns=fixed,  
  frame=none,  
  breaklines=true,
  showstringspaces=false,
  xleftmargin=0.2cm,
  %rangeprefix=//label, % to specify a certain range from a file
  %rangesuffix=;, % to be shown
  %includerangemarker=false,
  numbers=none
}


\newcommand{\inl}[1]{\lstinline{#1}}

\newcommand{\devicesoft}{\mbox{\textsc{Device-Soft}}\xspace}
\newcommand{\acsl}{\mbox{\textsf{ACSL}}\xspace}
\newcommand{\isoc}{\mbox{\textsf{C}}\xspace}
\newcommand{\macosx}{\mbox{Mac OS X}\xspace}
\newcommand{\framac}{\mbox{\textsf{Frama-C}}\xspace}
\newcommand{\jessie}{\mbox{\textsf{Jessie}}\xspace}
\newcommand{\why}{\mbox{\textsf{Why}}\xspace}
\newcommand{\whythree}{\mbox{\textsf{Why3}}\xspace}
\newcommand{\wpframac}{\mbox{\textsf{WP}}\xspace}
\newcommand{\wpqed}{\mbox{\textsf{Qed}}\xspace}
\newcommand{\altergo}{\mbox{\textsf{Alt-Ergo}}\xspace}
\newcommand{\coq}{\mbox{\textsf{Coq}}\xspace}
\newcommand{\cvc}{\mbox{\textsf{CVC4}}\xspace}

\newcommand{\cealist}{\mbox{\textsf{CEA LIST}}\xspace}
\newcommand{\inria}{\mbox{\textsf{INRIA}}\xspace}
\newcommand{\lri}{\mbox{\textsf{LRI}}\xspace}
\newcommand{\adacore}{\mbox{\textsf{AdaCore}}\xspace}
\newcommand{\sri}{\mbox{\textsf{SRI}}\xspace}
\newcommand{\ansi}{\mbox{\textsf{ANSI/ISO-C}}\xspace}
\newcommand{\xubuntu}{\mbox{\textsf{Xubuntu}}\xspace}

\newcommand{\cxx}{C\nolinebreak[4]\hspace{-.05em}\raisebox{.3ex}{\footnotesize\textbf{++}}\xspace}
% see http://www.parashift.com/c++-faq-lite/misc-environmental-issues.html#faq-40.2

\newcommand{\bitstream}{\texttt{Bitstream}\xspace}
\newcommand{\bitstreamread}{\texttt{Bitstream\_Read}\xspace}
\newcommand{\bitstreamwrite}{\texttt{Bitstream\_Write}\xspace}
\newcommand{\bitstreamnormal}{\inl{NormalBitstream}\xspace}



% to force correct separation of specific words
\hyphenation{non-nega-tive veri-fi-ca-tion simi-lar im-pera-tive com-pari-son col-labo-ra-tion speci-fi-ca-tion fa-mili-ar}

% to give output of the date in the form of month year 
\newcommand{\monthword}[1]{\ifcase#1\or January\or February\or March\or April\or
                                        May\or June\or July\or August\or
                                        September\or October\or November\or December\fi}
\newcommand{\todaymonth}{\monthword{\the\month} \the\year} 



\usepackage{float}
 
%\usepackage{chngcntr}
%\counterwithout{footnote}{chapter}

% need to appear as last \usepackage to avoid nonsensical warnings,
% cf. http://mrunix.de/forums/showthread.php?t=38657#8
\usepackage[pdftex]{hyperref}

\floatstyle{plain}
%\newfloat{listing}{thp}{lop1}[chapter]
\floatname{listing}{Listing}

%\newfloat{predfun}{thp}{lop2}[chapter]
\floatname{predfun}{Listing}


% share counters of all floats:
\makeatletter
\let\c@listing\c@figure
\let\c@predfun\c@figure
%\let\c@lemma\c@figure
\let\c@table\c@figure
\makeatother

% suppress hyperref warnings
\makeatletter
\def\Hy@Warning#1{}
\def\Hy@WarningNoLine#1{}
\makeatother

\renewcommand\floatpagefraction{.90}
\renewcommand\topfraction{.90}
\renewcommand\bottomfraction{.90}
\renewcommand\textfraction{.1}
\setlength{\unitlength}{1mm}

\setlength\topsep{1.5\baselineskip}  % larger space around theorem-like environments

\addtolength{\parskip}{0.5\baselineskip}
\parindent=0cm

\setcounter{tocdepth}{1}
\setcounter{secnumdepth}{2}

\begin{document}

\title{Specification Concepts for Packages of Subset 026 (Chapter 7)}
\author{Jens Gerlach}
\maketitle

\section{Predicates and Logic Functions of \inl{BitStream}}

The type \inl{Bitstream} models a bit stream and is given
by a byte array of \inl{size} elements that starts at \inl{addr}.
A bit stream also holds a \emph{bit position} that marks
the current position in the bit stream from where bits can be
read or to which bits can be written.

\begin{lstlisting}[style = acsl-block]
struct Bitstream
{
    uint8_t*  addr;
    uint32_t  size;
    uint32_t  bitpos;
};

typedef struct Bitstream Bitstream;
\end{lstlisting}

\begin{itemize}
\item
\inl{predicate Readable{L}(Bitstream* stream)}

This predicate is true if all bytes in the range
\inl{[stream->addr[0..size-1]} are valid for reading.

\item
\inl{predicate Writeable{L}(Bitstream* stream)}

This predicate is true if all bytes in the range
\inl{[stream->addr[0..size-1]} are valid for reading \emph{and} writing.

\item
\inl{predicate Invariant{L}(Bitstream* stream, integer length)}

This predicate is equivalent to the following condition

\begin{lstlisting}[style = acsl-block]
   8 * stream->size        <= UINT32_MAX   &&
   stream->bitpos + length <= UINT32_MAX;
\end{lstlisting}

\item
\inl{predicate Normal{L}(Bitstream* stream, integer length)}

This predicate is true if \inl{stream->bitpos + length <= 8 * stream->},
that is if at least \inl{length} bits can be read from or written to the
bitstream.

\end{itemize}

\section{Predicates and Logic Functions of Packages}


Let \inl{P} be the type of a package.

\begin{lstlisting}[style = acsl-block]
   struct P {
      ...
   };

   typedef struct P P;
\end{lstlisting}

We will use follwing \acsl predicates and logic functions with the following signatures
for the specification of decoder and encoder functions for \inl{P}.

\begin{itemize}
\item
\inl{logic integer MaxBitSize{L}(P* p)}

Returns the number of bits that might be read/written by \inl{P}.

\item
\inl{logic integer BitSize{L}(P* p)}

Returns the number of bits that have been read/written by \inl{P}.
This number is always less or equal than \inl{MaxBitSize}.

\item
\inl{predicate Invariant(P* p)}

This predicate is true if all type invariants of \inl{P}
are satisified.

\item
\inl{predicate ZeroInitialized(P* p)}

This predicate is true is all relevant elements of \inl{p} are
equal to zero.

\item
\inl{predicate EqualBits(Bitstream* stream, integer pos, P* p)}

This predicate is true if starting from the bit position
\inl{pos} in the bit stream \inl{stream} the following
\inl{BitSize} number of bits are equal to the corresponding bits
in the elements of \inl{p}.

\inl{predicate UpperBitsNotSet(P* p)}

This predicate is true if the appropriate upper (higher) bits
in the elements of \inl{p} are \emph{not} set.

\item
\inl{Separated(Bitstream* stream, P* p)}

This predicate is true if there is now aliasing between the elements of
\inl{p} and those of \inl{Bitstream}.

\end{itemize}

%\listoffixmes
%\tableofcontents
%\listof{predfun}{List of Logic Specifications}
%\listoffigures
%\listoftables


% jg:  the "phantomsection" is a hack to fix the wrong reference to bibliography
% see also http://sumanta679.wordpress.com/2009/05/15/latex-list-of-table-list-of-figures-and-bibliography-in-toc/
%\newpage
%\phantomsection \label{listoffig}
%\addcontentsline{toc}{chapter}{Bibliography}
%\bibliographystyle{unsrt}
%\bibliography{bibliography}

\end{document}

