\documentclass[parskip=half,paper=a4,DIV=12]{scrartcl}


\usepackage[utf8]{inputenc}
\usepackage{pgfplots}
\usepackage{amsmath}
\usepackage{amsthm}
\usepackage{tikz}
\usepackage{verbatim}
\usepackage{amssymb}
\usepackage{breqn}
\usepackage{graphicx}
\usepackage{color}
\usepackage{fancyhdr}
\usepackage{listings}
\usepackage{float}
\usepackage{stmaryrd}
\usepackage{cancel}

\setlength{\headheight}{22pt}
\pagestyle{fancy}

\lstdefinestyle{customc}{
  belowcaptionskip=1\baselineskip,
  breaklines=true,
  frame=L,
  xleftmargin=\parindent,
  language=C,
  showstringspaces=false,
  basicstyle=\footnotesize\ttfamily,
  keywordstyle=\bfseries\color{green!40!black},
  commentstyle=\itshape\color{purple!40!black},
  identifierstyle=\color{blue},
  stringstyle=\color{orange},
}

\lstdefinestyle{customasm}{
  belowcaptionskip=1\baselineskip,
  frame=L,
  xleftmargin=\parindent,
  language=[x86masm]Assembler,
  basicstyle=\footnotesize\ttfamily,
  commentstyle=\itshape\color{purple!40!black},
}
\lstset{escapechar=@,style=customc}



\title{Representation of C Structures in Bit Streams}

\author{Timon Lapawczyk \and Jens Gerlach}

\begin{document}

\maketitle

\section{Introduction}

We want to transmit simple C structures (packets) via bit streams.
The C structures consist of members of various (unsigned) integer types.
Moreover, it might happen that only a certain number of bits of a member
shall be transmitted. 

A packet types consists of $n$ member $x_1,\ldots, x_n$ and might look as follows:

\begin{lstlisting}[mathescape]
struct $\mathcal{A}$ {
	uint8_t  $x_{1}$      // # 8
	uint16_t $x_{2}$      // # 13
	uint8_t  $x_{3}$      // # 3
	uint32_t $x_{4}$      // # 23
};
\end{lstlisting}

If we choose to send\slash receive a packet of type $\mathcal{A}$, then
we only have to transmit $47 = 8 + 13 + 3 + 23$ bits instead of 64 bits.

In general, we denote the number of bits for a member $x_i$, with $1 \leq i \leq n$  by $a_i$.

\section{Decoder and Encoder}

To format the content of a packet into a bitstream with the minimum size, necessary to transfer all data, we use an Encoder.
It's inverse function is the Decoder, which reads the bits
from the bitstream and transforms them into a correct packet.
These functions consist of calls of peek (Decoder) and poke (Encoder) calls.
The number of peek and poke calls and their parameters depend on the number of variables and their lengths..
For the packet A the Decoder looks something like this:

\begin{lstlisting}[mathescape]
$\mathcal{A}$-Decoder (Bitstream stream, $\mathcal{A}$ a) {
	a->$x_{1}$ = peek (stream, 8);
	a->$x_{2}$ = peek (stream, 13);
	a->$x_{3}$ = peek (stream, 3);
	a->$x_{4}$ = peek (stream, 23);
}
\end{lstlisting}

\section{Bit Position}

The Decoder example is a very simplified version of an actual Decoder.
For the peek and poke calls we do actually need to set a bit position
that determines where in the stream the bits shall be read.
The bit position for each call depends on the bit position and length of the previous call.
For the packet $\mathcal{A}$ the bit position count up as follows:\\
variable   $a_{1}$, $a_{2}$, $a_{3}$, $a_{4}$\\
length      8, 13,  3, 23\\
bit position      0,  8, 21, 24\\
For each variable $x_{k}$ we can calculate the bit position $S_{k}$ as follows:
\begin{align*}
	&S_{k} = \sum_{i=1}^{k-1} a_{i}\\
	&S_{k+1} = S_{k} + a_{k}
\end{align*}

\section{Dynamic packets}

C structures of dynamic packets don't look any different from those of the previously introduced (static) packets.
The only difference is that, whether or not the value of a variable is of importance, may depend on the value of another variable.	
For example in our $\mathcal{A}$ packet, the value of $x_{3}$ may only be significant, if $x_{2}$ has a certain value.
Since some data may be useless we don't want to waste any bandwidth on our transmission media and therefore only transfer the needed data.
In the example we only want to transfer $x_{3}$ if $x_{2}$ has a certain value.
Our Decoder function looks something like that:

\begin{lstlisting}[mathescape]
$\mathcal{A}$-Decoder (Bitstream stream, $\mathcal{A}$ a) {
	a->$x_{1}$ = peek (stream, 8);
	a->$x_{2}$ = peek (stream, 13);
	if ($x_{2}$ == $\ldots$) {
		a->$x_{3}$ = peek (stream, 3);
	}
	a->$x_{4}$ = peek (stream, 23);
}
\end{lstlisting}

But how do we calculate the bit position for $x_{4}$, since it also depend on the value of $x_{2}$.

variable   $x_{1}$, $x_{2}$, $x_{3}$, $x_{4}$\\
length        8, 13, 3, 23\\
bit position\\
$x_{2} = \ldots  0, 8, 21, 24$\\
$x_{2} \neq \ldots  0, 8, 21, 21$\\

Maybe with the same if/else block:

\begin{lstlisting}[mathescape]
if ( $x_{2}$ == $\ldots$) {
	$S_{4}$ = $S_{3}$ + $a_{3}$
} else {
	$S_{4}$ = $S_{3}$
}
\end{lstlisting}

\section{Valid Array}

A different approach to the determination of bit position in dynamic packets is to create a boolean array, that holds a valid flag for each variable.
\begin{align*}
	&x_{1}, x_{2}, x_{3}, \ldots , x_{n}\\
	&v_{1}, v_{2}, v_{3}, \ldots , v_{n}	
\end{align*}
For a given packet $\mathcal{A}$ with a variable sequence $x_{1}, x_{2}, x_{3}, \ldots , x_{n}$ each $v_{i}$ determines whether or not variable $x_{i}$ was transferred.
Assuming we have a $V$-array, then we don't have to use the if-block.
Neither for the peek and poke nor for the bit position determination.
For each $x_{i}$ with $1\leq i\leq n$ we have the length of bits to read $a_{i}$ and the boolean $v_{i}$ that tells us whether or not the variable was transferred.
the Decoder look something like that:
\begin{lstlisting}[mathescape]
$\mathcal{A}$-Decoder (Bitstream stream, $\mathcal{A}$ a) {
	a->$x_{1}$ = peek (stream, 8 * $v_{1}$);
	a->$x_{2}$ = peek (stream, 13 * $v_{2}$);
	a->$x_{3}$ = peek (stream, 3 * $v_{3}$);
	a->$x_{4}$ = peek (stream, 23 * $v_{4}$);
}
\end{lstlisting}

\end{document}
