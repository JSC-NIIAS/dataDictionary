\documentclass[parskip=half,paper=a4,DIV=13]{scrartcl}


\usepackage[utf8]{inputenc}
\usepackage{pgfplots}
\usepackage{amsmath}
\usepackage{amsthm}
\usepackage{tikz}
\usepackage{verbatim}
\usepackage{amssymb}
\usepackage{breqn}
\usepackage{graphicx}
\usepackage{color}
\usepackage{fancyhdr}
\usepackage{listings}
\usepackage{float}
\usepackage{stmaryrd}
\usepackage{cancel}

\setlength{\headheight}{22pt}
\pagestyle{fancy}

\lstdefinestyle{customc}{
  belowcaptionskip=1\baselineskip,
  breaklines=true,
  frame=L,
  xleftmargin=\parindent,
  language=C,
  showstringspaces=false,
  basicstyle=\footnotesize\ttfamily,
  keywordstyle=\bfseries\color{green!40!black},
  commentstyle=\itshape\color{purple!40!black},
  identifierstyle=\color{blue},
  stringstyle=\color{orange},
}

\lstdefinestyle{customasm}{
  belowcaptionskip=1\baselineskip,
  frame=L,
  xleftmargin=\parindent,
  language=[x86masm]Assembler,
  basicstyle=\footnotesize\ttfamily,
  commentstyle=\itshape\color{purple!40!black},
}
\lstset{escapechar=@,style=customc}



\title{Representation of C Structures in Bit Streams}

\author{Timon Lapawczyk \and Jens Gerlach}

\begin{document}

\maketitle

\section{Introduction}

We want to transmit simple C structures (packets) via bit streams.
The C structures consist of members of various (unsigned) integer types.
Moreover, it might happen that only a certain number of bits of a member
shall be transmitted. 

A packet types consists of $n$ member $x_1,\ldots, x_n$ and looks as follows:

\begin{lstlisting}[mathescape]
struct $\mathcal{A}$ {
	uint8_t  $x_{1}$      // # 8
	uint16_t $x_{2}$      // # 13
	uint8_t  $x_{3}$      // # 3
	uint32_t $x_{4}$      // # 23
};
\end{lstlisting}

If we choose to send\slash receive a packet of type $\mathcal{A}$, then
we only have to transmit $47 = 8 + 13 + 3 + 23$ bits instead of 64 bits.

In general, we denote the number of bits for a member $x_i$, with $1 \leq i \leq n$  by $a_i$.

\section{Decoder and Encoder}

For each packet type there is a \emph{decoder} and a \emph{encoder} function.
\begin{itemize}
\item
The decoder extracts the specified bits from the bit stream and stores them
in the individual packet members.
\item
The encoder, conversely, extracts the bits from the individual members
and writes them into the bit stream.
\end{itemize}

The extraction function of the decoder is referred to as \emph{peek},
whereas the extraction function of the encoder is referred to as \emph{poke}.
The number of calls to peek and poke and the parameters depend on the number
of variables and the required bit lengths.

\pagebreak
For the packet $\mathcal{A}$ the decoder looks like this:

\begin{lstlisting}[mathescape]
$\mathcal{A}$ Decoder(Bitstream stream) {
        $\mathcal{A}$ a;
	a.$x_{1}$ = peek (stream, 8);
	a.$x_{2}$ = peek (stream, 13);
	a.$x_{3}$ = peek (stream, 3);
	a.$x_{4}$ = peek (stream, 23);
}
\end{lstlisting}

\section{Bit Position}

The decoder example above is simplified in the fact, that for each peek call we do actually
need to set a bit position that determines where in the stream the bits shall be read.
The bit position for each call depends on the bit position and length of the previous call.
For the packet $\mathcal{A}$ the bit position counts up as follows:\\
\begin{center}
\begin{tabular}{r | cccc}
	variable &$x_{1}$ &$x_{2}$ &$x_{3}$ &$x_{4}$\\
	length &8 &13 &3 &23\\
	bit position &0 &8 &21 &24
\end{tabular}
\end{center}
For each variable $x_{k}$ we can calculate the bit position $S_{k}$ as follows:
\begin{align*}
	S_{k} = \sum_{i=1}^{k-1} a_{i}\\
\end{align*}
As described above, the bit position for each peek call is determined by
the bit position of the previous call and the length of the previous peek and can therefore expressed by
\begin{align*}
	S_{k+1} = S_{k} + a_{k}.
\end{align*}

The following decoder example includes the bit positions and each peek call reads like this:
\begin{center}
	peek (stream, bit position, length);
\end{center}

\begin{lstlisting}[mathescape]
$\mathcal{A}$ Decoder(Bitstream stream) {
        $\mathcal{A}$ a;
	a.$x_{1}$ = peek (stream, 0,  8);
	a.$x_{2}$ = peek (stream, 8,  13);
	a.$x_{3}$ = peek (stream, 21, 3);
	a.$x_{4}$ = peek (stream, 24, 23);
}
\end{lstlisting}

\section{Dynamic packets}

C structures of dynamic packets don't look any different from the previously introduced (static) packets.
The only difference is that, whether or not the value of a variable is of importance, may depend
on the value of a previously read variable.

For example in the $\mathcal{A}$ packet, the value of $x_{3}$ might only be significant,
if $x_{2}$ has the value $y$.

Just like with the unnecessary unset bits, we don't want to transfer data of no imortance.
In the example we only want to transfer $x_{3}$ if $x_{2}$ has the value $y$.
The decoder function then looks like that:

\begin{lstlisting}[mathescape]
$\mathcal{A}$ Decoder(Bitstream stream) {
        $\mathcal{A}$ a;
	a.$x_{1}$ = peek (stream, 8);
	a.$x_{2}$ = peek (stream, 13);
	if ($x_{2}$ == $y$) {
		a.$x_{3}$ = peek (stream, 3);
	}
	a.$x_{4}$ = peek (stream, 23);
}
\end{lstlisting}

The example does not include any bit positons yet. 
In section 3 we defined the bit position of the peek call for each $x_{k}$ with $S_{k}$.
$S_{k}$ depends on the bit position of $x_{k-1}$ and the length of $x_{k-1}$,
but what if thet length of $x_{k-1}$ is variable?
Since this gets a bit confusing lets bring up the table with the bit positions again. 

\begin{center}
\begin{tabular}{l r | cccc}
	\multicolumn{2}{r|}{variable} &$x_{1}$ &$x_{2}$ &$x_{3}$ &$x_{4}$\\
	\multicolumn{2}{r|}{length} &8 &13 &3 &23\\
	\multirow{2}{*}{bit position} &$x_{2} = y$  &0 &8 &21 &\bf{24}\\
	 &$x_{2} \neq y$  &0 &8 &21 &\bf{21}
\end{tabular}
\end{center}

As we can see, the bit position of $x_{4}$ depends on the value of $x_{2}$, so we can use
the same if-/else block as in the decoder above to determine it.

\begin{lstlisting}[mathescape]
if ( $x_{2}$ == $y$) {
	$S_{4}$ = $S_{3}$ + $a_{3}$
} else {
	$S_{4}$ = $S_{3}$
}
\end{lstlisting}

\section{Valid Array}

A different approach to the determination of bit positions in dynamic packets is to use
a boolean array, that holds a valid flag for each variable.
\begin{center}
\begin{tabular}{ccccc}
	$x_{1}$ &$x_{2}$ &$x_{3}$ &$\ldots$ &$x_{n}$\\
	$v_{1}$ &$v_{2}$ &$v_{3}$ &$\ldots$ &$v_{n}$	
\end{tabular}
\end{center}

For a given packet $\mathcal{A}$ with a variable sequence $x_{1}, x_{2}, x_{3}, \ldots , x_{n}$
each $v_{i}$ determines whether or not variable $x_{i}$ was transferred.

To use the array to create a much cleaner decoder, we make use of the feature of the
peek function, to also accept the length 0.

For each $x_{i}$ with $1\leq i\leq n$ we have the length of bits to read $a_{i}$
and the boolean $v_{i}$ that tells us whether or not the variable was transferred.

With these bits of information we can calculate  $a_{i}^{\prime}$ as the actual length
for each $x_{i}$ which may also be zero.
\begin{align*}
	a_{i}^{\prime} = a_{i} \cdot v_{i}.
\end{align*}

The new decoder (without bit positions) looks like this:
\begin{lstlisting}[mathescape]
$\mathcal{A}$ Decoder (Bitstream stream) {
	$\mathcal{A}$ a;
	a.$x_{1}$ = peek (stream, $a_{1}^{\prime}$);
	a.$x_{2}$ = peek (stream, $a_{2}^{\prime}$);
	a.$x_{3}$ = peek (stream, $a_{3}^{\prime}$);
	a.$x_{4}$ = peek (stream, $a_{4}^{\prime}$);
}
\end{lstlisting}

Since we found a way to build the decoder without the if-/else blocks
we may also think of a way to deternime the actual bit position $S_{k}^{\prime}$ for each variable.
The bit position of the peek call for $x_{i}$ is the same bit position as $S_{i-1}^{\prime}$ plus the the actual length $a_{i-1}^{\prime}$.
\begin{align*}
	S_{k+1}^{\prime} = S_{k} + a_{k}^{\prime}
\end{align*}
From that we can write the actual bit position for each $k$ with $1 \leq k \leq n$ like
\begin{align*}
	S_{k}^{\prime} = \sum_{i=1}^{k-1} a_{i}^{\prime}.
\end{align*}

With the actual bit positions the decoder looks like this:
\begin{lstlisting}[mathescape]
$\mathcal{A}$ Decoder (Bitstream stream) {
	$\mathcal{A}$ a;
	a.$x_{1}$ = peek (stream, $S_{1}^{\prime}$, $a_{1}^{\prime}$);
	a.$x_{2}$ = peek (stream, $S_{2}^{\prime}$, $a_{2}^{\prime}$);
	a.$x_{3}$ = peek (stream, $S_{3}^{\prime}$, $a_{3}^{\prime}$);
	a.$x_{4}$ = peek (stream, $S_{4}^{\prime}$, $a_{4}^{\prime}$);
}
\end{lstlisting}

\section{Ideas}

\begin{lstlisting}[mathescape]
$\mathcal{A}$ Decoder (Bitstream stream) {
	$\mathcal{A}$ a;
	$S^{\prime} = pos$;
	a.$x_{1}$ = peek (stream, $S^{\prime}$, $a_{1}^{\prime}$);
	$S^{\prime}$ += $a_{1}^{\prime}$;
	a.$x_{2}$ = peek (stream, $S^{\prime}$, $a_{2}^{\prime}$);
	$S^{\prime}$ += $a_{2}^{\prime}$;
	a.$x_{3}$ = peek (stream, $S^{\prime}$, $a_{3}^{\prime}$);
	$S^{\prime}$ += $a_{3}^{\prime}$;
	a.$x_{4}$ = peek (stream, $S^{\prime}$, $a_{4}^{\prime}$);
	$S^{\prime}$ += $a_{4}^{\prime}$;
}
\end{lstlisting}

\begin{lstlisting}[mathescape]
$\mathcal{A}$ Valid (V-Array *v, $\mathcal{A}$ *a) {
	v[0] = 1;	# $x_{1}$
	v[1] = 1;	# $x_{2}$
	if (v[0] == y) {
		v[2] = 1;	# $x_{3}$
	} else {
		v[2] = 0;	# $x_{3}$
	}
	v[3] = 1;	# $x_{4}$
}
\end{lstlisting}

\begin{lstlisting}[mathescape]
$\mathcal{A}$ Decoder (Bitstream stream) {
	$\mathcal{A}$ a;
	$\mathcal{A}$ V-Array v;
	$S^{\prime}$ = $pos$;
	a.$x_{1}$ = peek (stream, $S^{\prime}$, $a_{1}\cdot v[0]$);
	$\mathcal{A}$ Valid (&v, &a);
	$S^{\prime}$ += $a_{1}^{\prime}$;
	a.$x_{2}$ = peek (stream, $S^{\prime}$, $a_{2}\cdot v[1]$);
	$\mathcal{A}$ Valid (&v, &a);
	$S^{\prime}$ += $a_{2}^{\prime}$;
	a.$x_{3}$ = peek (stream, $S^{\prime}$, $a_{3}\cdot v[2]$);
	$\mathcal{A}$ Valid (&v, &a);
	$S^{\prime}$ += $a_{3}^{\prime}$;
	a.$x_{4}$ = peek (stream, $S^{\prime}$, $a_{4}\cdot v[3]$);
	$\mathcal{A}$ Valid (&v, &a);
	$S^{\prime}$ += $a_{4}^{\prime}$;
}
\end{lstlisting}

\end{document}
